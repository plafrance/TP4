\documentclass[12pt,pdftex,oneside]{article}
% Mes ajouts pour les accents
\usepackage[utf8]{inputenc}%
\usepackage[T1]{fontenc}%
\usepackage[french]{babel}%

\include{mes_macros}
%\include{graphicx}

\renewcommand{\theenumii}{\roman{enumii}}%

\begin{document}

  \section {Classe Belligérant}

  Un belligérant capable de combattre à mains nues (et en pagne).

  \begin{itemize}
  \item Propriétés : 
    \begin{enumerate}
    \item nom : Le nom du belligérant.
          \begin{itemize}
          \item Type : str
          \item Accès : lecture seule
          \end{itemize}
    \item pts\_vies : Le nombre de points de vie courant du belligérant.
          \begin{itemize}
          \item Type : int
          \item Accès : accès complet
          \end{itemize}
    \item pts\_vies\_max : Le nombre de points de vie initial du belligérant.
          \begin{itemize}
          \item Type : int
          \item Accès : lecture seule
          \end{itemize}
    \item défense : Le facteur de défense du belligérant. Plus ce facteur est
      élevé, plus le belligérant pourra résister aux attaques.
          \begin{itemize}
          \item Type : int
          \item Accès : lecture seule
          \end{itemize}
    \item force : Le facteur de force du belligérant. Plus ce facteur est
      élevé, plus les attaques du belligérant seront efficaces.
          \begin{itemize}
          \item Type : int
          \item Accès : lecture seule
          \end{itemize}
    \item \_items : Ensemble d'Items que possède le Belligérant.
          \begin{itemize}
          \item Type : Ensemble d'Items
          \item Accès : lecture seule
          \end{itemize}

    \end{enumerate}

  \item Constructeur : 

    \begin{enumerate}
    \item \_\_init\_\_ : Initialise un belligérant. Outre le nom, fournit en
paramètre, les valeurs de force, défense et de points de vie sont calculés de la
façon suivante :
        \begin{itemize}
          \item force : 1 dé12
          \item défense : 1 dé12
          \item pts\_vie : 2 dé12 + 20
        \end{itemize}
    \end{enumerate}

  \item Méthodes : 

    \begin{enumerate}
    \item nom : \_\_str\_\_ : Fournit une chaîne de caractère représentant le belligérant.
      \begin{itemize}
      \item Retour : Une chaîne représentant le belligérant de la forme «nom
        (F:\emph{force} D:\emph{défense} V:\emph{pts\_vie})
          \begin{itemize}
          \item Type de retour: str
          \end{itemize}
      \end{itemize}

    \item attaquer : Calcule le coefficient d'attaque d'un assaut  par la formule.
      \begin{itemize}
      \item Retour : Le coefficient d'attaque d'un assaut calculé.
        : \emph{force} + 1 dé12
          \begin{itemize}
          \item Type de retour: int
          \end{itemize}
      \end{itemize}

    \item parer : Calcule le coefficient de parade lors d'un assaut.
      \begin{itemize}
      \item Retour : Le coefficient de parade lors d'un assaut calculé par la
        formule : \emph{défense} + 2 dé6
          \begin{itemize}
          \item Type de retour: int
          \end{itemize}
      \end{itemize}

    \item subir\_dégâts : Après avoir reçu une attaque, cette méthode calcule les
      dégâts subis par le belligérant et les déduit de ses points de vie selon
      la formule : dégâts = impact.
      \begin{itemize}
      \item Paramètres : 
        \begin{enumerate}
        \item impact (int) : La force d'impact de l'attaque reçue.
        \end{enumerate}
      \item Retour : Le nombre de points de vie réellement enlevé au Belligérant.
      \end{itemize}

    \item est\_mort : détermine si un belligérant est mort.
      \begin{itemize}
      \item Retour : Vrai si et seulement si pts\_vie est 0 ou moins.
          \begin{itemize}
          \item Type de retour: bool
          \end{itemize}
      \end{itemize}

    \item prendre\_item : Ajoute un item à ceux que possède le Belligérant. Le
      poids total de ses items ne peut excéder 5Kg par point de force.
      \begin{itemize}
      \item Paramètres : 
        \begin{enumerate}
        \item un\_item (Item) : l'item à ajouter aux items du belligérant.
        \end{enumerate}
      \item Assertions : 
        \begin{enumerate}
            \item Le poids total des items (incluant celui ajouté) ne dépasse
              pas 5kg par point de force du Belligérant.
            Belligérants
            \begin{itemize}
              \item Message : «L'item est trop lourd pour être ajouté au Belligérant».
            \end{itemize}
        \end{enumerate}
      \end{itemize}

    \item jeter\_item : Retire un item de la liste d'items détenus par le Belligérant.
      \begin{itemize}
      \item Paramètres : 
        \begin{enumerate}
        \item un\_item (Item) : l'item à enlever. 
        \end{enumerate}
      \end{itemize}
      \item Assertions : 
        \begin{enumerate}
            \item L'item existe dans la liste.
            \begin{itemize}
              \item Message : «L'item n'existe pas».
            \end{itemize}

        \end{enumerate}
    \end{enumerate}
  \end{itemize}

  \section {Classe Équipe}

  Regroupe les belligérants d'une même équipe (d'un même joueur).

  \begin{itemize}
  \item Propriétés : 
    \begin{enumerate}
    \item nom : nom de l'équipe
          \begin{itemize}
          \item Type : str
          \item Accès : Lecture seule
          \end{itemize}

    \end{enumerate}

  \item Constructeur : 

  \begin{enumerate}
  \item \_\_init\_\_ : Initialise une équipe avec un nom mais sans belligérants
    immédiatement.
    \begin{itemize}
    \item Paramètres : 
      \begin{enumerate}
      \item nom (str) : Le nom de la nouvelle équipe.
      \end{enumerate}
    \end{itemize}

  \end{enumerate}

  \item Méthodes : 

    \begin{enumerate}
    \item \_\_len\_\_ : Retourne le nombre d'éléments dans l'équipe.
      \begin{itemize}
      \item Retour : Le nombre de belligérants dans l'équipe
          \begin{itemize}
          \item Type de retour: int
          \end{itemize}
      \end{itemize}

    \item ajouter\_belligérant : Ajoute un belligérant à l'équipe
      \begin{itemize}
      \item Paramètres : 
        \begin{enumerate}
        \item un\_belligérant : le nouveau belligérant à ajouter à l'équipe.
          \begin{itemize}
          \item Type : Belligérant
          \end{itemize}
        \end{enumerate}
      \end{itemize}

    \item belligérant : Accesseur des belligérants
      \begin{itemize}
      \item Paramètres : 
        \begin{enumerate}
        \item indice : numéro du belligérant à retourner
          \begin{itemize}
          \item Type : int
          \item Valeur par défaut : None
          \end{itemize}
        \item nom : nom du belligérant à retourner
          \begin{itemize}
          \item Type : str
          \item Valeur par défaut : None
          \end{itemize}
        \end{enumerate}
      \item Retour : Si \emph{indice} n'est pas None, retourne le belligérant numéro
        \emph{indice} dans la liste, sinon, retourne le belligérant dont le nom
        correspond à \emph{nom}. Si \emph{indice} et \emph{nom} sont None, retourne None.
          \begin{itemize}
          \item Type de retour: Belligérant
          \end{itemize}
      \item Assertions : 
        \begin{enumerate}
            \item \emph{indice} est None ou représente un élément existant de la liste de
            Belligérants
            \begin{itemize}
              \item Message : «indice \{\emph{indice}\} invalide»
            \end{itemize}
            \item \emph{nom} est None ou représente le nom d'un élément existant de la liste de
            Belligérants
            \begin{itemize}
              \item Message : «nom '\{\emph{indice}\}' est invalide»
            \end{itemize}
        \end{enumerate}
      \end{itemize}

    \item est\_éliminée : Retourne Vrai si l'équipe est éliminée.
      \begin{itemize}
      \item Retour : Vrai si et seulement si tous les combattants de l'équipe
        sont morts.
          \begin{itemize}
          \item Type de retour: bool
          \end{itemize}
      \end{itemize}

    \end{enumerate}
  \end{itemize}

  \section {Classe Guerrier}

  Un belligérant de type Guerrier. Il peut porter une armure et manier une arme.

  \begin{itemize}
  \item Hérite de : Belligérant
  \item Propriétés : 
    \begin{enumerate}
    \item armure : L'armure que porte actuellement le guerrier
          \begin{itemize}
          \item Type : Armure
          \item Accès : complet
          \end{itemize}
    \item arme : L'arme qu'utilise actuellement le guerrier. Le Guerrier ne peut
      utiliser une arme dont la classe dépasse la sienne.
          \begin{itemize}
          \item Type : Arme
          \item Accès : complet
          \end{itemize}

    \end{enumerate}

  \item Constructeur : 

  \begin{enumerate}
  \item \_\_init\_\_ : Initialise un Guerrier, d'abord sans arme ni armure. Ses
    facteurs de force et de défense obtiennent un boni d'un dé12 par rapport au
    Belligérant et ses points de vie de 2xDé20.
    \begin{itemize}
    \item Paramètres : 
      \begin{enumerate}
      \item un\_nom (str) : Le nom du nouveau Guerrier
      \end{enumerate}
    \end{itemize}

  \end{enumerate}

  \item Méthodes : 

    \begin{enumerate}
    \item calculer\_dégâts : Surdéfinition de la méthode
      Belligérant.calculer\_dégâts(). En plus du calcul des dégâts comme pour
      n'importe quel belligérant, les dégâts réels sont réduits de le moitié de
      la classe de l'armure portée par le guerrier.
      \begin{itemize}
      \item Paramètres : 
        \begin{enumerate}
        \item impact (int) : La force d'impact de l'attaque reçue.
        \end{enumerate}
      \end{itemize}
      \item Retour : Le nombre de points de vie qui doivent être retirés au
        Guerrier suite à l'attaque.
          \begin{itemize}
          \item Type de retour: int
          \end{itemize}

    \item attaquer : Surdéfinition de la méthode Belligérant.attaquer. Le
      coefficient d'attaque du belligérant est multiplié par la classe de l'arme
      qu'il utilise au moment de l'attaque.
      \begin{itemize}
      \item Retour : Le coefficient d'attaque d'un assaut.
          \begin{itemize}
          \item Type de retour: int
          \end{itemize}
      \end{itemize}

    \item arme : Mutateur de l'arme.
      \begin{itemize}
      \item Paramètres : 
        \begin{enumerate}
        \item une\_arme (Arme) : L'arme qui doit être utilisée par le
          Guerrier. Le Guerrier ne peut utiliser une arme dont la classe dépasse
          la sienne.
        \end{enumerate}
      \item Assertions : 
        \begin{enumerate}
            \item La classe de l'arme n'est pas plus grand que celle du Guerrier.
            Belligérants
            \begin{itemize}
              \item Message : «Le Guerrier ne peut utiliser une arme dont la
                classe est plus grande que la sienne.»
            \end{itemize}
        \end{enumerate}

    \item subir\_dégâts : Surdéfinition de la méthode
      Belligérant.subir\_dégâts(). Soustrait au Guerrier le nombre de points de
      vie calculé par calculer\_dégâts. L'armure subit ensuite autant
      d'usure qu'un cinquième de l'impact.
      \begin{itemize}
      \item Paramètres : 
        \begin{enumerate}
        \item impact (int) : La force d'impact de l'attaque reçue.
        \end{enumerate}
      \end{itemize}

      \end{itemize}

    \end{enumerate}
  \end{itemize}


  \section {Classe Mage}

  Un belligérant de type mage capable de jeter des sorts.

  \begin{itemize}
  \item Hérite de : Belligérant
  \item Propriétés : 
    \begin{enumerate}
    \item puissance : La puissance du mage. Pour lancer un sort, il ne peut
      jeter que des sorts dont la classe est inférieure ou égale à sa puissance.
          \begin{itemize}
          \item Type : int
          \item Accès : complet
          \end{itemize}
    \item mana : La quantité d'«énergie» magique que possède le Mage. Il ne peut
      lancer de sort qui demande plus de mana que la quantité qu'il possède.
          \begin{itemize}
          \item Type : int
          \item Accès : complet
          \end{itemize}
    \end{enumerate}

  \item Constructeur : 

  \begin{enumerate}
  \item \_\_init\_\_ : Initialise un Mage. Sa puissance est donnée par un dé6 et
    sa mana par 2xDé20. Initialement, il ne possède aucun sort.
    \begin{itemize}
    \item Paramètres : 
      \begin{enumerate}
      \item param (un\_nom) : Le nom du Mage.
      \end{enumerate}
    \end{itemize}
  \end{enumerate}

  \item Méthodes : 

    \begin{enumerate}
    \item jeter\_sort : Jete un sort vers une cible. Pour réussir à lancer son sort, la
      différence entre la puissance du mage et la classe du sort doit être plus
      grande que le lancé d'un dé6. Dans tous les cas, la mana du mage est diminuée d'autant qu'il est requis par le sort.
      \begin{itemize}
      \item Paramètres : 
        \begin{enumerate}
        \item sort (Sort) : Le sort qui doit être jeté par le Mage
        \item cible (Belligérant) : Le belligérant vers qui le sort est jeté.
        \end{enumerate}
      \item Assertions : 
        \begin{enumerate}
        \item La puissance du Mage doit être au moins aussi grande que le classe
          de \emph{sort}
          \begin{itemize}
          \item Message : «Puissance (\emph{puissance}) < Sort.classe
            (\emph{sorte.classe})        
          \end{itemize}
        \item La mana du Mage doit être au moins aussi grande que la mana requise
          par le \emph{sort}
          \begin{itemize}
          \item Message : «Mana (\emph{mana}) < Sort.mana\_requise (\emph{sorte.mana\_requise})
          \end{itemize}
        \end{enumerate}
      \end{itemize}

    \item parer : Calcule le coefficient de parade lors d'un assaut. Puisque le
      Mage est plus agile que le Belligérant moyen, son coefficient la valeur arrondie de 10\% par
      point de puissance de plus que celui calculé par Belligérant.parer().
      \begin{itemize}
      \item Retour : Le coefficient de parade lors d'un assaut.
          \begin{itemize}
          \item Type de retour: int
          \end{itemize}
      \end{itemize}

    \item ajouter\_sort : Ajoute un sort à la liste des sorts connus par le Mage.
      \begin{itemize}
      \item Paramètres : 
        \begin{enumerate}
        \item un\_sort (sort) : Le nouveau Sort à ajouter au Mage.
        \end{enumerate}
      \item Assertions : 
        \begin{enumerate}
        \item Le mage ne doit pas déjà posséder ce sort
          \begin{itemize}
          \item Message : «Le Sort (\emph{un\_sort}) existe déjà»
          \end{itemize}
        \end{enumerate}
      \end{itemize}

    \end{enumerate}
  \end{itemize}

  \section {Classe Sort}

  Classe abstraite représentant un sort qui peut être jeté par un Mage.

  \begin{itemize}
  \item Propriétés : 
    \begin{enumerate}
    \item mana\_requise : quantité de mana requise pour jeter ce sort
          \begin{itemize}
          \item Type : int
          \item Accès : lecture seule
          \end{itemize}
    \item classe : la classe du sort, représente la difficulté à jeter ce sort.
          \begin{itemize}
          \item Type : int
          \item Accès : lecture seule
          \end{itemize}

    \end{enumerate}

  \item Méthodes : 

    \begin{enumerate}
    \item \_\_str\_\_ : Fournit une représentation en chaîne de caractères du Sort.
      \begin{itemize}
      \item Retour : Une représentation du Sort sous la forme «classe \emph{classe}; mana requise : \emph{mana\_requise}»
      \end{itemize}
    \item activer : Méthode abstraite. Active le sort qui agit alors sur sa cible.
      \begin{itemize}
      \item Paramètres : 
        \begin{enumerate}
        \item cible (Belligérant) : Le belligérant ciblé par le sort.
        \end{enumerate}
      \end{itemize}

    \end{enumerate}

  \end{itemize}


  \section {Classe SortDartDeFeu}

  Sort qui lance un dart de feu vers un belligérant.

  \begin{itemize}
  \item Hérite de : Sort


  \item Constructeur : 

  \begin{enumerate}
  \item \_\_init\_\_ : Initialise le Sort avec sa classe (1) et sa mana (2).

  \end{enumerate}

  \item Méthodes : 

    \begin{enumerate}
    \item \_\_str\_\_ : Fournit une représentation en chaîne de caractères du Sort.
      \begin{itemize}
      \item Retour : Une représentation du Sort sous la forme «Dart de feu: classe \emph{classe}; mana requise : \emph{mana\_requise}»
      \end{itemize}
    \item activer : Active le sort qui agit alors sur sa cible. Il porte une
      attaque de puissance de 2xDé12. La cible peut parer le sort comme
      pour une attaque physique.
      \begin{itemize}
      \item Paramètres : 
        \begin{enumerate}
        \item cible (Belligérant) : Le belligérant ciblé par le sort.
        \end{enumerate}
      \end{itemize}

    \end{enumerate}
  \end{itemize}

  \section {Classe SortGuérison}

  Sort qui permet d'augmenter le nombre de points de vie d'un belligérant.

  \begin{itemize}
  \item Hérite de : Sort


  \item Constructeur : 

  \begin{enumerate}
  \item \_\_init\_\_ : Initialise le Sort avec sa classe (1) et sa mana (4).

  \end{enumerate}

  \item Méthodes : 

    \begin{enumerate}
    \item \_\_str\_\_ : Fournit une représentation en chaîne de caractères du Sort.
      \begin{itemize}
      \item Retour : Une représentation du Sort sous la forme «Guérison : classe \emph{classe}; mana requise : \emph{mana\_requise}»
      \end{itemize}
    \item activer : Active le sort qui agit alors sur sa cible. Il porte redonne
      1 dé20 de points de vie à sa cible seulement si elle n'est pas morte, sinon, elle ne fait rien.
      \begin{itemize}
      \item Paramètres : 
        \begin{enumerate}
        \item cible (Belligérant) : Le belligérant ciblé par le sort.
        \end{enumerate}
      \end{itemize}

    \end{enumerate}
  \end{itemize}

  \section {Classe SortProtection}

  Sort qui augmente la protection d'un belligérant.

  \begin{itemize}
  \item Hérite de : Sort


  \item Constructeur : 

  \begin{enumerate}
  \item \_\_init\_\_ : Initialise le Sort avec sa classe (2) et sa mana (8).

  \end{enumerate}

  \item Méthodes : 

    \begin{enumerate}
    \item \_\_str\_\_ : Fournit une représentation en chaîne de caractères du Sort.
      \begin{itemize}
      \item Retour : Une représentation du Sort sous la forme «Protection : classe \emph{classe}; mana requise : \emph{mana\_requise}»
      \end{itemize}
    \item activer : Active le sort qui agit alors sur sa cible. Il augmente la
      défense du belligérant de 50\%.
      \begin{itemize}
      \item Paramètres : 
        \begin{enumerate}
        \item cible (Belligérant) : Le belligérant ciblé par le sort.
        \end{enumerate}
      \end{itemize}

    \end{enumerate}
  \end{itemize}

  \section {Classe SortRésurrection}

  Description

  \begin{itemize}
  \item Hérite de : Sort

  \item Constructeur : 

  \begin{enumerate}
  \item \_\_init\_\_ : Initialise le Sort avec sa classe (4) et sa mana (15).

  \end{enumerate}

  \item Méthodes : 

    \begin{enumerate}
    \item \_\_str\_\_ : Fournit une représentation en chaîne de caractères du Sort.
      \begin{itemize}
      \item Retour : Une représentation du Sort sous la forme «Résurrection : classe \emph{classe}; mana requise : \emph{mana\_requise}»
      \end{itemize}
    \item activer : Active le sort qui agit alors sur sa cible. Il redonne à sa
      cible 2xDé20 points de vie.
      \begin{itemize}
      \item Paramètres : 
        \begin{enumerate}
        \item cible (Belligérant) : Le belligérant ciblé par le sort.
        \end{enumerate}
      \end{itemize}

    \end{enumerate}
  \end{itemize}

  \section {Classe Arme}

  Classe abstraite représentant une arme générique.

  \begin{itemize}
  \item Hérite de : Item
  \item Propriétés : 
    \begin{enumerate}
    \item classe : La classe de l'arme. Plus la classe est élevée, plus elle est destructrice.
          \begin{itemize}
          \item Type : int
          \item Accès : lecture seule
          \end{itemize}
    \item bonus : Le bonus d'attaque accordé par l'arme. Cette propriété est
      calculée par les sous-classes.
          \begin{itemize}
          \item Type : int
          \item Accès : lecture seule
          \end{itemize}

    \end{enumerate}

  \item Constructeur : 

  \begin{enumerate}
  \item \_\_init\_\_ : Initialise un Arme avec sa classe.
    \begin{itemize}
    \item Paramètres : 
      \begin{enumerate}
      \item une\_classe (int) : La classe de l'arme.
      \end{enumerate}
      \item Assertions : 
        \begin{enumerate}
        \item La classe ne peut être négative
          \begin{itemize}
          \item Message : «La classe (\emph{une\_classe}) est invalide»
          \end{itemize}
        \end{enumerate}
    \end{itemize}

  \end{enumerate}

  \item Méthodes : 

    \begin{enumerate}
    \item \_\_str\_\_ : Méthode abstraite qui retourne une représentation en chaîne de caractère de l'arme
      sous la forme «Type d'arme (classe)».
      \begin{itemize}
      \item Retour :  une représentation en chaîne de caractère de l'arme
      sous la forme «Type d'arme (classe)».
          \begin{itemize}
          \item Type de retour: str
          \end{itemize}
      \end{itemize}
    \item bonus : Méthode abstraite qui calcule le bonus accordé par cette arme
      pour une attaque.
      \begin{itemize}
      \item Retour : Le bonus accordé par cette arme.
          \begin{itemize}
          \item Type de retour: int
          \end{itemize}
      \end{itemize}

    \end{enumerate}
  \end{itemize}

  \section {Classe Armure}

  Classe abstraite représentant une armure générique.

  \begin{itemize}
  \item Hérite de : Item
  \item Propriétés : 
    \begin{enumerate}
    \item classe : La classe de l'armure. Plus la classe est élevée, plus elle
      protège celui qui la porte.
          \begin{itemize}
          \item Type : int
          \item Accès : lecture seule
          \end{itemize}
    \item bonus : Le bonus de défense accordé par l'arme. Cette propriété est
      calculée par les sous-classes.
          \begin{itemize}
          \item Type : int
          \item Accès : lecture seule
          \end{itemize}
    \item usure : Le pourcentage d'usure de l'armure. Initialement à 0, elle
      augent progressivement jusqu'à 100; l'armure est alors rendue inutile.
          \begin{itemize}
          \item Type : int
          \item Accès : complet
          \end{itemize}

    \end{enumerate}

  \item Constructeur : 

  \begin{enumerate}
  \item \_\_init\_\_ : Initialise un Armure avec sa classe.
    \begin{itemize}
    \item Paramètres : 
      \begin{enumerate}
      \item une\_classe (int) : La classe de l'armure.
      \end{enumerate}
      \item Assertions : 
        \begin{enumerate}
        \item La classe ne peut être négative
          \begin{itemize}
          \item Message : «La classe (\emph{une\_classe}) est invalide»
          \end{itemize}
        \end{enumerate}
    \end{itemize}

  \end{enumerate}

  \item Méthodes : 

    \begin{enumerate}
    \item \_\_str\_\_ : Méthode abstraite qui retourne une représentation en chaîne de caractère de l'armure.
      sous la forme «Type d'armure (classe)».
      \begin{itemize}
      \item Retour :  une représentation en chaîne de caractère de l'armure
      sous la forme «Type d'armure (classe)».
          \begin{itemize}
          \item Type de retour: str
          \end{itemize}
      \end{itemize}
    \item bonus : Méthode abstraite qui calcule le bonus accordé par cette armure
      pour une attaque.
      \begin{itemize}
      \item Retour : Le bonus accordé par cette armure.
          \begin{itemize}
          \item Type de retour: int
          \end{itemize}
      \end{itemize}

    \end{enumerate}
  \end{itemize}

  \section {Classe Dé}

  Simule un dé avec un nombre variable de faces.

  \begin{itemize}

  \item Méthodes : 

    \begin{enumerate}
    \item lancer : Méthode de classe qui simule un lancer de dé.
      \begin{itemize}
      \item Paramètres : 
        \begin{enumerate}
        \item faces (int) : le nombre de faces du dé à lancer.
          \begin{itemize}
          \item Valeur par défaut : 6
          \end{itemize}
      \end{enumerate}
      \item Retour : Un nombre au hasard entre 1 et \emph{faces} inclusivement.
          \begin{itemize}
          \item Type de retour: int
          \end{itemize}
      \item Assertions : 
        \begin{enumerate}
        \item \emph{faces} doit être > 1
          \begin{itemize}
          \item Message : «Le nombre de faces doit être > 1»
          \end{itemize}
        \end{enumerate}
      \end{itemize}

    \end{enumerate}
  \end{itemize}


  \section {Classe Item}

  Classe abstraite représentant un item générique pouvant être acquis et porté
  par un Belligérant.

  \begin{itemize}
  \item Propriétés : 
    \begin{enumerate}
    \item nom : poids
          \begin{itemize}
          \item Type : int
          \item Accès : lecture seule
          \end{itemize}

    \end{enumerate}

  \item Constructeur : 

  \begin{enumerate}
  \item \_\_init\_\_ : Initialise l'item avec son poids.
    \begin{itemize}
    \item Paramètres : 
      \begin{enumerate}
      \item un\_poids (int) : le poids de l'item en grammes.
      \end{enumerate}
    \end{itemize}

  \end{enumerate}

  \item Méthodes : 

    \begin{enumerate}
    \item \_\_str\_\_ : Donne une représentation en chaîne de caractères de l'item sous la forme «p=\emph{poids}».
      \begin{itemize}
      \item Retour : La représentation en chaîne de caractères de l'item.
          \begin{itemize}
          \item Type de retour: str
          \end{itemize}
      \end{itemize}

    \end{enumerate}
  \end{itemize}
  \section {Classe Potion}

  Classe abstraite représentant une potion buvable par un Belligérant.

  \begin{itemize}
  \item Hérite de : Item
  \item Propriétés : 
    \begin{enumerate}
    \item parts : Le nombre de parts que contient la fiole de potions.
          \begin{itemize}
          \item Type : int
          \item Accès : lecture seule
          \end{itemize}

    \end{enumerate}

  \item Constructeur : 

  \begin{enumerate}
  \item \_\_init\_\_ : Initialise la potion avec son nombre de parts.
    \begin{itemize}
    \item Paramètres : 
      \begin{enumerate}
      \item nb\_part (int) : Le nombre de parts contenues dans la fiole de potion.
      \end{enumerate}
    \item Assertions : 
      \begin{enumerate}
      \item Le nombre de parts ne peut être < 1.
        \begin{itemize}
          \item Message : «nb\_parts ne peut être < 1.»
        \end{itemize}
      \end{enumerate}
    \end{itemize}
  \end{enumerate}

  \item Méthodes : 

    \begin{enumerate}
  \item faire\_boire : Fait boire la potion par un belligérant. Décrémente de
    un le nombre de parts restantes. L'effet de la potion est exécutée par la méthode \emph{activer} des sous-classes.
    \begin{itemize}
    \item Paramètres : 
      \begin{enumerate}
      \item cible (Belligérant) : Le Belligérant qui doit boire la potion.
      \end{enumerate}
      \item Assertions : 
        \begin{enumerate}
        \item Le nombre de parts ne peut être < 1.
          \begin{itemize}
          \item Message : «nb\_parts ne peut être < 1.»
          \end{itemize}
        \end{enumerate}
    \end{itemize}

    \item activer : méthode abstraite qui exécute l'effet de la potion sur la cible
      \begin{itemize}
    \item Paramètres : 
      \begin{enumerate}
      \item cible (Belligérant) : Le Belligérant qui doit boire la potion.
      \end{enumerate}
      \item Assertions : 
        \begin{enumerate}
        \item Le nombre de parts ne peut être < 1.
          \begin{itemize}
          \item Message : «nb\_parts ne peut être < 1.»
          \end{itemize}
        \end{enumerate}
    \end{itemize}

    \end{enumerate}
  \end{itemize}


  \section {Classe PotionGuérison}

  Description

  \begin{itemize}
  \item Hérite de : Potion
  \item Propriétés : 
    \begin{enumerate}
    \item pts\_de\_vie\_par\_part : Le nombre de pts de vie redonnés par cette
      potion pour chaque part bue.
          \begin{itemize}
          \item Type : int
          \item Accès : lecture seule
          \end{itemize}

    \end{enumerate}

  \item Constructeur : 

  \begin{enumerate}
  \item \_\_init\_\_ : Initialise la PotionGuérison avec son nombre de parts et
    de points de vie par part.
    \begin{itemize}
    \item Paramètres : 
      \begin{enumerate}
      \item nb\_part (int) : Le nombre de parts contenues dans la fiole de potion.
      \item pts\_de\_vie\_par\_part : Le nombre de pts de vie redonnés par cette
      potion pour chaque part bue.
      \end{enumerate}
      \item Assertions : 
        \begin{enumerate}
        \item Le nombre de parts ne peut être < 1.
          \begin{itemize}
          \item Message : «nb\_parts ne peut être < 1»
          \end{itemize}
        \end{enumerate}
    \end{itemize}
  \end{enumerate}
  \end{itemize}

  \section {Classe Joueur}

  Un joueur qui contrôle une équipe de Belligérants.

  \begin{itemize}
  \item Attributs :
    \item contrôle : Un Contrôle par lequel le joueur peut interagir avec le jeu.
          \begin{itemize}
          \item Type : Contrôle
          \end{itemize}

  \item Propriétés : 
    \begin{enumerate}
    \item nom : Le peudonyme du joueur
          \begin{itemize}
          \item Type : str
          \item Accès : lecture seule
          \end{itemize}
    \item équipe : L'équipe que contrôle le joueur
          \begin{itemize}
          \item Type : Équipe
          \item Accès : lecture seule
          \end{itemize}
    \item actions\_par\_tour : Le nombre d'actions que peut faire le joueur à
      chaque tour.
          \begin{itemize}
          \item Type : int
          \item Accès : complet
          \end{itemize}

    \end{enumerate}

  \item Constructeur : 

  \begin{enumerate}
  \item \_\_init\_\_ : Initialise le Joueur avec un nom sans espaces au début et
    à la fin et un Contrôle.
    \begin{itemize}
    \item Paramètres : 
      \begin{enumerate}
      \item un\_nom (str) : Le pseudonyme du Joueur
      \item un\_contrôle (Contrôle) : Le Contrôle par lequel le joueur interagira
        avec le jeu.
      \end{enumerate}
      \item Assertions : 
        \begin{enumerate}
        \item Le nom ne peut être une chaîne vide.
          \begin{itemize}
          \item Message : «nom invalide»
          \end{itemize}
        \end{enumerate}
    \end{itemize}

  \end{enumerate}

  \item Méthodes : 

    \begin{enumerate}
    \item \_\_str\_\_ : Donne une représentation en chaîne de caractères du Joueur.
      \begin{itemize}
      \item Retour : Le nom du joueur.
          \begin{itemize}
          \item Type de retour: str
          \end{itemize}
      \end{itemize}

    \item créer\_belligérant : Crée un belligérant et l'ajoute à l'équipe du
      joueur. Utilise le contrôle pour permettre au joueur de choisir le type et
      le nom de son belligérant.

    \item jouer : utilise le contrôle pour permettre au joueur de jouer son
      tour. Un tour comporte les étapes suivantes :
      \begin{enumerate}
        \item Choisir un belligérant dans son équipe.
        \item Choisir une action que ce belligérant doit effectuer.
        \item Si l'action est une attaque, choisir son adversaire.
        \item Si l'action concerne un item, choisir l'item
        \item Si l'action est jeter un sort, choisir le sort et la cible
        \item Effectuer l'action sur la cible.
      \end{enumerate}
      Ces étapes sont répétées tant que le nombre d'actions par tour du joueur
      n'est pas atteint ou qu'il n'y pas choisit de passer son tour.

    \item choisir\_belligérant : Utilise le contrôle pour permettre au joueur de
      choisir un belligérant dans sa propre équipe.
      \begin{itemize}
      \item Retour : Le belligérant sélectionné.
          \begin{itemize}
          \item Type de retour: Belligérant
          \end{itemize}
      \end{itemize}

    \item choisir\_action : Utilise le contrôle pour permettre au joueur de
      choisir sa prochaine action.
      \begin{enumerate}
      \item Paramètres : 
        \begin{enumerate}
        \item un\_acteur (Belligérant) : Le belligérant qui doit faire
          l'action. La liste des actions proposées sera construite en fonction
          du type et des capacités de ce Belligérant.
      \item Retour : L'action choisie par le joueur
          \begin{itemize}
          \item Type de retour: Action
          \end{itemize}
        \end{enumerate}
    \end{enumerate}

    \item choisir\_cible : Utilise le contrôle pour permettre au joueur de
      choisir un belligérant ciblé par une action. Le belligérant peut être de
      son équipe ou de celle d'un adversaire.
      \begin{itemize}
      \item Paramètres : 
        \begin{enumerate}
        \item les\_équipes (Liste d'Équipes) : La liste de toutes les équipes
          parmi lesquelles le joueur peut choisir sa cible.
        \end{enumerate}
      \item Retour : Le belligérant sélectionné.
          \begin{itemize}
          \item Type de retour: Belligérant
          \end{itemize}
      \end{itemize}

    \end{enumerate}
  \end{itemize}


  \section {Classe Contrôle}

  Classe abstraite représentant un Contrôle capable de faire interagir le joueur
  avec le jeu. Le Contrôle permet au joueur de faire sélectionner 
  plusieurs types d'éléments. Des méthodes abstraites permettent de communiquer
  directement avec le joueur (par l'écran et le clavier, par exemple).

  \begin{itemize}

  \item Méthodes : 

    \begin{enumerate}
    \item choisir\_Objet
      \begin{itemize}
      \item Paramètres : 
        \begin{enumerate}
        \item choix (Liste d'Objets) : Liste des objets parmi lesquels
          choisir. (Par exemple, une liste de Belligérants ou d'Items.)
        \end{enumerate}
      \item Retour : L'objet sélectionné.
          \begin{itemize}
          \item Type de retour: Objet
          \end{itemize}
      \end{itemize}
    \end{enumerate}
  \end{itemize}


  \section {Classe ContrôleXXX}

  Classe représentant un Contrôle capable de faire interagir le joueur
  avec le jeu. Le Contrôle permet au joueur de faire sélectionner 
  plusieurs types d'éléments. Des méthodes abstraites permettent de communiquer
  directement avec le joueur (par l'écran et le clavier, par exemple).

  \begin{itemize}

  \item Hérite de : Contrôle
  \item Méthodes : 
    \begin{enumerate}

    \item choisir
      \begin{itemize}
      \item Paramètres : 
        \begin{enumerate}
        \item choix (Liste de str) : Liste des choix offerts au joueurs.
        \end{enumerate}
      \item Retour : Le numéro du choix sélectionné.
          \begin{itemize}
          \item Type de retour: int
          \end{itemize}
      \end{itemize}

    \item afficher\_message : Permet d'afficher un message au joueur sans
      attendre de réponse de sa part.
      \begin{itemize}
      \item Paramètres : 
        \begin{enumerate}
        \item un\_message (str) : Le message à afficher.
        \item confirmation (bool) : détermine si la méthode doit attendre que le joueur
          ait confirmé la lecture du message pour continuer.
          \begin{itemize}
          \item Valeur par défaut : False
          \end{itemize}
        \end{enumerate}
      \end{itemize}

    \item saisir : Permet au joueur de saisir un texte.
      \begin{itemize}
      \item Paramètres : 
        \begin{enumerate}
        \item question (str) : La question posée au joueur. Aucune question si
          None.
          \begin{itemize}
          \item Valeur par défaut : None
          \end{itemize}
        \end{enumerate}
      \item Retour : Le texte saisi par le joueur.
          \begin{itemize}
          \item Type de retour: str.
          \end{itemize}
      \end{itemize}

    \end{enumerate}
  \end{itemize}


  \section {Classe Partie}

  Une partie de Tournoi Pythonesque 4 entre deux ou plusieurs joueurs.

  \begin{itemize}

  \item Attributs : 
    \begin{enumerate}
    \item \_joueurs : Les joueurs de cette partie.
          \begin{itemize}
          \item Type : Liste de joueurs
          \end{itemize}
    \item \_tour : Le numéro du tour courant. 0 pendant la préparation du jeu
          \begin{itemize}
          \item Type : int
          \end{itemize}
    \item \_items\_épars : Les items laissés sur le champ de bataille. Lorsqu'un
      belligérant meurt, ses items sont remis ici.
          \begin{itemize}
          \item Type : Liste d'Items.
          \end{itemize}

  \item Constructeur : 

  \begin{enumerate}
  \item \_\_init\_\_ : Initialise la partie.
    \begin{itemize}
    \item Paramètres : 
      \begin{enumerate}
      \item les\_joueurs (Liste de Joueurs) : Liste de tous les participants à
        la partie.
      \end{enumerate}
      \item Assertions : 
        \begin{enumerate}
        \item La liste doit contenir au moins deux joueurs
          \begin{itemize}
          \item Message : «Il n'y a pas suffisamment de joueurs.»
          \end{itemize}
        \end{enumerate}
    \end{itemize}

  \end{enumerate}

  \item Méthodes : 

    \begin{enumerate}
    \item populer\_équipes : Permet de créer tous les belligérants de toutes les
      équipes. Tour à tour, chaque joueur est invité à créer son belligérant. Le
      joueur qui commence est choisi au hasard.

    \item créer\_belligérant : Permet de créer un belligérant. Selon le mode de
      partie choisie, les belligérants sont créés ici au hasard ou en demandant au
      joueur de choisir certaines caractéristiques (par sa méthode
      créer\_belligérant).
    \item Paramètres : 
      \begin{enumerate}
      \item un\_joueur (Joueur) : Le joueur pour lequel créer le belligérant. Si
        None, le belligérant est créé au hasard.
      \end{enumerate}
    \item Retour : Le belligérant créé.

    \item démarrer\_partie : Joue la partie jusqu'à ce que toutes les équipes
      sauf une soient éliminées. Le premier joueur à jouer est choisi au
      hasard. À tour de rôle, les joueurs jouent leur tour. Dès qu'une équipe
      est éliminée, le joueur devient spectateur.

    \item jouer\_tour : Joue un tour de jeu. Tous les joueurs, à tour de rôle
      jouent leur tour.

    \item répartir\_items : Après la création des belligérants, mais avant le
      début de la partie, les items épars sont répartis entre les
      belligérants. Le joueur qui commence est choisi au hasard puis tous, à
      tour de rôle sélectionnent un item et un belligérant pour le
      prendre. Jusqu'à ce qu'il n'y ait plus d'items ou que tous les
      belligérants aient atteint la limite de ce qu'ils peuvent porter.

    \end{enumerate}

    \end{enumerate}
  \end{itemize}
  \end{itemize}
\end{document}

