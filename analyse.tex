\documentclass[12pt,pdftex,oneside]{article}
% Mes ajouts pour les accents
\usepackage[utf8]{inputenc}%
\usepackage[T1]{fontenc}%
\usepackage[french]{babel}%

\include{mes_macros}
%\include{graphicx}

\renewcommand{\theenumii}{\roman{enumii}}%

\begin{document}

  \section {Classe Belligérant}

  Un belligérant capable de combattre à mains nues (et en pagne).

  \begin{itemize}
  \item Propriétés : 
    \begin{enumerate}
    \item nom : Le nom du belligérant.
          \begin{itemize}
          \item Type : str
          \item Accès : lecture seule
          \end{itemize}
    \item pts\_vies : Le nombre de points de vie du belligérant.
          \begin{itemize}
          \item Type : int
          \item Accès : accès complet
          \end{itemize}
    \item défense : Le facteur de défense du belligérant. Plus ce facteur est
      élevé, plus le belligérant pourra résister aux attaques.
          \begin{itemize}
          \item Type : int
          \item Accès : lecture seule
          \end{itemize}
    \item force : Le facteur de force du belligérant. Plus ce facteur est
      élevé, plus les attaques du belligérant seront efficaces.
          \begin{itemize}
          \item Type : int
          \item Accès : lecture seule
          \end{itemize}

    \end{enumerate}

  \item Constructeur : 

    \begin{enumerate}
    \item \_\_init\_\_ : Instancie un belligérant. Outre le nom, fournit en
paramètre, les valeurs de force, défense et de points de vie sont calculés de la
façon suivante :
        \begin{itemize}
          \item force : 1 dé12
          \item défense : 1 dé12
          \item pts\_vie : 2 dé12 + 20
        \end{itemize}

      \begin{itemize}
      \item Paramètres : 
        \begin{enumerate}
        \item param : description
          \begin{itemize}
          \item Type : type
          \end{itemize}
        \end{enumerate}
      \item Assertions : 
        \begin{enumerate}
          \item Condition : condition de l'assertion
          \item Message : message de l'assertion
        \end{enumerate}
      \end{itemize}
    \end{enumerate}

  \item Méthodes : 

    \begin{enumerate}
    \item nom : \_\_str\_\_ : Fournit une chaîne de caractère représentant le belligérant.
      \begin{itemize}
      \item Retour : Une chaîne représentant le belligérant de la forme «nom
        (F:\emph{force} D:\emph{défense} V:\emph{pts\_vie})
          \begin{itemize}
          \item Type de retour: str
          \end{itemize}
      \end{itemize}

    \item attaquer : Calcule le coefficient d'attaque d'un assaut.
      \begin{itemize}
      \item Retour : Le coefficient d'attaque d'un assaut calculé par la formule
        : \emph{force} + 1 dé12
          \begin{itemize}
          \item Type de retour: int
          \end{itemize}
      \end{itemize}

    \item parer : Calcule le coefficient de parade lors d'un assaut.
      \begin{itemize}
      \item Retour : Le coefficient de parade lors d'un assaut calculé par la
        formule : \emph{défense} + 2 dé6
          \begin{itemize}
          \item Type de retour: int
          \end{itemize}
      \end{itemize}

    \item subir\_dégâts : Après avoir reçu une attaque, cette méthode calcule les
      dégâts subis par le belligérant et les déduit de ses points de vie.
      \begin{itemize}
      \item Paramètres : 
        \begin{enumerate}
        \item impact : La force d'impact de l'attaque reçue.
          \begin{itemize}
          \item Type : int
          \end{itemize}
        \end{enumerate}
      \end{itemize}

    \item est\_mort : détermine si un belligérant est mort.
      \begin{itemize}
      \item Retour : Vrai si et seulement si pts\_vie est 0 ou moins.
          \begin{itemize}
          \item Type de retour: bool
          \end{itemize}
      \end{itemize}

  \end{itemize}

  \end{itemize}
\end{document}

